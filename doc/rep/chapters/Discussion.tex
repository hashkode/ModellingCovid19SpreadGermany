\section{Discussion}
The new approach to pandemic modeling based on networked entities represented by adjacency matrices that are programmatically calculated based on official data from government agencies shows good performance for recovery and prediction of the pandemic activity on the nation level.

As was shown, the small number of parameters depicted in \autoref{tbl:seirSettings} is sufficient to tailor the framework. At the same time, different data can be easily included for identification and simulation with the networked SEIR model by extending the behavior/policy factors $\psi$. Both aspects allow to consider other data sources for COVID-19 simulation or modify the framework to apply it to other epidemics.

Still, \autoref{fig:compAggrCombWave0} and the plots in \autoref{sec:variationsOfSEIRSettings} show that the approach presented herein does not model the COVID-19 outbreak either completely or reliably. This is partly caused by neglected effects\footnote{Vaccinations are not considered and mandate the introduction of a new compartment with transitions from susceptible to vaccinated. Additionally, new studies show that neither recovered nor fully vaccinated people exhibit permanent protection from infections. Hence, a SEIRS model as proposed \cite{bjornstadSEIRSModelInfectious2020} with a transition from removed to susceptible should be investigated.}, but also likely influenced by underreporting of cases causing a wrong estimation of the actual pandemic levels. It is further affected by time-varying test policy and number of tests executed--and subsequently positive test ratios--as well as official COVID-19 legislation and varying spread characteristics of the different SARS-CoV-2 virus strains--though the spread supporting factors are still under investigation.

\subsection{Sensitivity Analysis}
An extensive sensitivity analysis was not performed due to the large size of the data sets and the extensive set of parameters and settings. Nevertheless, a number of different scopes of analysis and their impact is presented in \autoref{sec:variationsOfSEIRSettings}.
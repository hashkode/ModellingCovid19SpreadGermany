% %%%%%%%%%%%%%%%%%%%%%%%%%%%%%%%%%%%%%%%%%%%%%%%%%%%%%%%%%%%%%%
% NOTE: Using this is optional
% 	Nonetheless, feel free to include this file and adjust the 
% 	examples below as needed
%   
% Questions, feedback and improvements:
%	-> https://git.lsr.ei.tum.de/students/student-templates/issues
% ATTENTION:
% 	- Keep in mind that thif file uses the \vec and \mat commands
% 	 	which are defined in customize.tex!
%	- add glsaddall if you want all of the elements in gloss.aux
%		being printed into your list of acronyms.
% Further reading:
%	https://ctan.org/pkg/glossaries?lang=en
%%%%%%%%%%%%%%%%%%%%%%%%%%%%%%%%%%%%%%%%%%%%%%%%%%%%%%%%%%%%%%

% include packages
\usepackage{setspace}
\usepackage{filecontents}
% for sharelatex, the indexing using xindy:
% 	http://xindy.sourceforge.net/doc/faq-1.html#ss1.2
% is not available, thus the adjustments below ar necessary
%% Adjust Flag for Sharelatex
\newif\ifShareLatex 
%\ShareLatextrue            	% if you work in sharelatex: https://sharelatex.tum.de
\ShareLatexfalse          		% if you work locally

\ifShareLatex
    \usepackage[acronym, style=alttree, shortcuts, toc=true, nomain, nonumberlist]{glossaries}
    \renewcommand{\makeglossaries}{\makenoidxglossaries}
    \renewcommand{\printglossary}{\printnoidxglossary}
\else
    \usepackage[acronym,style=alttree, toc=true, shortcuts, xindy, nomain, nonumberlist]{glossaries}
    \RequirePackage[xindy]{imakeidx}
\fi

%%%%%%%%%%%%%%%%%%%%%%%%%%%%%%%%%%
% DEFINE HEADINGS AND CATEGORIES %
%%%%%%%%%%%%%%%%%%%%%%%%%%%%%%%%%%
% preferable use a command, to adjust the Capitalization 
% for the header section of fancyhdr automatically
\newcommand{\Symbols}{List of Symbols}
\newcommand{\Notation}{Notation}
\newglossary{symbols}{sym}{sbl}{\Symbols}
\newglossary{notation}{not}{nt}{\Notation}
% set width of first row
\glssetwidest{THISWIDE}     % adjust length as needed

%%%%%%%%%%%%%%%%%%%%%%%%%%%%%
% DEFINE ACRONYMS/GLOSSARY	%
%%%%%%%%%%%%%%%%%%%%%%%%%%%%%
\makeglossaries % don't remove this
\begin{filecontents}{gloss.aux}
	%===========%
	% ACRONYMS	%
	%===========%
	\newacronym{MPC}{MPC}{model-predictive control}
	\newacronym{BIBO}{BIBO}{bounded-input bounded-output}
	\newacronym{HRC}{HRC}{Human-Robot Collaboration}
	%============%
	% SYMBOLS	%
	%===========%
	\newglossaryentry{control}{type=symbols,
		sort={control},
		name={\ensuremath{\vec{u}}},
		description={control input vector}
	}
    \newglossaryentry{uk}{type=symbols,
		sort={control},
		name={\ensuremath{\vec{u}_k}},
		description={control input vector with time step}
	}
    \newglossaryentry{xk}{type=symbols,
		sort={state},
		name={\ensuremath{\vec{x}_k}},
		description={state vector with time step}
	}
	%============%
	% NOTATION	%
	%===========%
	\newglossaryentry{vector}{type=notation,
		sort={vector},
		name={\ensuremath{\vec{x}_n}},
		description={$n$-dimensional vector named $x$}
	}	
	\newglossaryentry{matrix}{type=notation,
		sort={vector-matrix},
		name={\ensuremath{\Mat{x}_{m\times n}}},
		plural={matrices},
		user1={Mat},
		description={\ensuremath{m\times n} dimensional Matrix  named \ensuremath{X}}
	}	
\end{filecontents}
\loadglsentries{gloss.aux}

%%%% Add GLOSSARIES at end of thesis
\newcommand{\AddMyGloss}{
	\renewcommand{\glsglossarymark}[1]{}
   	\printglossary[type=acronym]
	\markboth{\MakeUppercase{acronyms}}{\MakeUppercase{acronyms}}
  	\ifdefined\Symbols
		\printglossary[type=symbols, nogroupskip]
		\markboth{\MakeUppercase{\Symbols}}{\MakeUppercase{\Symbols}}
	\fi
	\ifdefined\Notation
		\printglossary[type=notation, nogroupskip]
		\markboth{\MakeUppercase{\Notation}}{\MakeUppercase{\Notation}}
	\fi
}


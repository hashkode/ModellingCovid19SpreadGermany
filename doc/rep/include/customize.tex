%%%%%%%%%%%%%%%%%%%%%%%%%%%%%%%%%%%%%%%%%%%%%%%%%%%%%%%%%%%%%%
% CUSTOMIZING Tex-File for students to be adjusted as needed %
%%%%%%%%%%%%%%%%%%%%%%%%%%%%%%%%%%%%%%%%%%%%%%%%%%%%%%%%%%%%%%
%%%%%%%%%%%%%%%%%%%%%
% CUSTOM PACKAGES	%
%%%%%%%%%%%%%%%%%%%%%
\usepackage{tikz}
\usepackage{pgfplots}
\usepackage{textgreek}
\usepackage{dsfont}
\usepackage{caption}
\usepackage{subcaption}
\usepackage{flushend}

% \usepackage{IEEEtrantools}

%%%%%%%%%%%%%%%%%%%%%
% CUSTOM COMMANDS	%
%%%%%%%%%%%%%%%%%%%%%
% e.g. Math conventions as vectors, Matrices, sets
\renewcommand{\vec}[1]{\mathbf{\MakeLowercase{#1}}}
\newcommand{\Mat}[1]{\mathbf{\MakeUppercase{#1}}}
\newcommand{\set}[1]{\boldsymbol{#1}}

\makeatletter
\newcounter{manualsubequation}
\renewcommand{\themanualsubequation}{\alph{manualsubequation}}
\newcommand{\startsubequation}{%
  \setcounter{manualsubequation}{0}%
  \refstepcounter{equation}\ltx@label{manualsubeq\theequation}%
  \xdef\labelfor@subeq{manualsubeq\theequation}%
}
\newcommand{\tagsubequation}{%
  \stepcounter{manualsubequation}%
  \tag{\ref{\labelfor@subeq}\themanualsubequation}%
}
\let\subequationlabel\ltx@label
\makeatother

\DeclareMathOperator*{\argmin}{arg\,min}
\DeclareMathSymbol{\shortminus}{\mathbin}{AMSa}{"39}

% e.g. use differenc layers in tikz
\pgfdeclarelayer{background}
\pgfdeclarelayer{nodelayer}
\pgfdeclarelayer{edgelayer}
\pgfdeclarelayer{foreground}
\pgfsetlayers{background,edgelayer,nodelayer,main,foreground}
%%%%%%%%%%%%%%%%%%%%%
% HYPHENATIONS 		%
%%%%%%%%%%%%%%%%%%%%%
\hyphenation{Lya-pu-nov}
